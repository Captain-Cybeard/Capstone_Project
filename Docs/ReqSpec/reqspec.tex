\documentclass{article}
\usepackage[margin=1in]{geometry}
\usepackage{fancyhdr}
\usepackage{graphicx}
\usepackage{vhistory}
\usepackage[parfill]{parskip}
\graphicspath{{./}}

% Set fancy looking header/footer and move page number to the right
\pagestyle{fancy}
\fancyhead{}
\fancyfoot{}
\fancyfoot[R]{\thepage}

\begin{document}

    % For large document with titlepage:
    \pagenumbering{gobble}
    \begin{titlepage}
    \begin{center}
        \vspace*{1cm}

        \Huge
        \textbf{Requirements Specification}

        \vspace{.5cm}
        \LARGE
        Captain Cybeard: Neil Before Us

        \vspace{1cm}

        \textbf{Ryan Breitenfeldt \textbar\ Noah Farris\\ Trevor Surface \textbar\ Kyle Thomas}

        \vspace{.2cm}
        \Large
        September 27, 2019

        \vspace{2cm}
        \includegraphics[scale=1]{logo}

        \vfill

        Washington State University Tri-Cities\\
        CptS 421 Software Design Project 1

    \end{center}
\end{titlepage}



    \tableofcontents
    \listoffigures

    \newpage
    \begin{versionhistory}
        \vhEntry{0.6}{10.29.2019}{KT}{Edits}
        \vhEntry{0.5}{10.28.2019}{NF}{Background}
        \vhEntry{0.4}{10.28.2019}{RB}{Introduction}
        \vhEntry{0.3}{10.15.2019}{TS}{Completed Overview}
        \vhEntry{0.2}{10.10.2019}{RB NF TS KT}{Filled in Environment \& Operation sections}
        \vhEntry{0.1}{09.27.2019}{KT}{Document Creation}
    \end{versionhistory}
    \newpage

    \pagenumbering{arabic}
    \section{Introduction}
    The requirements specification document is to go over the requirements for the Virtual Machine file Downloader for Cypherpath's Resiliency Platform.
    This document will cover how the application will provide a solution for users to easily upload their Virtual Machine files to the Resiliency Platform and also
    go in depth on how the users will interact with the application through a web interface. This document is intended to be understood and agreed
    upon by both the customer (Cypherpath) and the software development team (Captian CyBeard).

    Subsequent sections will include some background information on Cypherpath and their Resiliency Platform, an overview of the application and what it
    is supposed to accomplish the environment the application will execute in, and finally, how users will interact the application.

%%%    This application is to give the user the ability to download their virtual machines for upload to Cypherpath’s resiliency platform.
%%%    It will be a Django web application in Python3. The user will be able to get the virtual machines they have in the VMware cloud and AWS cloud.
%%%    It should detect what service is from the URL supplied by the user. The user will be given a selectable list of virtual machines on
%%%    their desired cloud service for them to chose from. The user will select from the list and download the wanted virtual machine.
%%%    The virtual machine will then be downloaded to the local computer. The requirements from Cypherpath are detailed here in how Cypherpath desires to
%%%    accomplish this program. This program will later be expanded by Cypherpath to be integrated into their software.     
    

    \section{Background}
	The Cypherpatch Resiliency Platform tool is a product that gives customers the ability to upload virtual machine files and network configurations into 
    self-contained digital environments, enabling customers to quickly recover from ransomware attacks. Right now, customers have to manually download their Virtual Machine files from
    the various platforms they are hosted on and then upload those to the Resiliency Platform. Cypherpath would like a web application that simplifies file downloading for users.

	This project aims to give Cypherpath users a simple means of retrieving Virtual Machine files from different websites and plaftorms, while making it easy to 
	then upload those files to Cypherpath's Resiliency Platform, providing more value to the customer in the process.


    \section{Overview}
    The purpose of this application is to allow users to download the vm file from an online Virtual Machine (VM), to local file storage.
    The application itself will operate in the web browser using Python3 and Django. Upon starting the application the user would
    provide a URL of their VM. Processing the URL the user is then asked to provide credentials to validate they are the owner of
    the VM. After completing the verification, the application will then use the API of the client the user has their VM stored on
    to access their folders. The User will be shown their directory structure and choose to download files locally to their desktop.
    Upon accepting to download locally the application will use another API to download the files local to their VM.

    The application will be written with Django and Python3, including the requests libraries for API calls, SAML for Authentication
    and standard libraries.

    The application only allows users to view the files from VM's they own, and download those file locally to the machine they 
    are running the application through. The users will have the ability to provide either a VMware URL to download from, or an
    Amazon Web Services (AWS) VM URL. The application will be designed with modularity to allow further 
    development with different VM service providers.  


    \section{Environment}

    %%%%% DELETE ME LATER %%%%%%%%%%%%%%%%%%%%%%%%%%%%%
    % sections: web app
    %           vm apis (VMware, AWS)
    %           authentication modules
    %           local hard drive? (for saving files)
    %%%%%%%%%%%%%%%%%%%%%%%%%%%%%%%%%%%%%%%%%%%%%%%%%%%
    The environment that the software will preside in consists of modules to interact with VMware, 
    AWS, other virtual machine platforms and will also be interacting with authentication modules for those
    various platforms. The user will interact with the API's of these cloud services and authentication
    methods through a Django Web Application. Lastly, the environment will consist of the application storing the selected
    virtual machines onto the user's local machine.

    %%%%%%%%%%%%%%% DIAGRAM STILL NEEDS TO BE FIXED!!! %%%%%%%%%%%%%%%%%%%%%%%%%%
    \begin{figure}[h]
    \includegraphics[scale=.7]{diagram}
        \caption{A visual representation of the applications environment.}
    \end{figure}
    %%%%%%%%%%%%%%% DIAGRAM STILL NEEDS TO BE FIXED!!! %%%%%%%%%%%%%%%%%%%%%%%%%%

%%%%%        \subsection{VMWare (VSphere)}
%%%%%
%%%%%            \subsubsection{API'S}
%%%%%            Use VMWare API that looks at file structs.
%%%%%
%%%%%%            Example URL:
%%%%%
%%%%%            \subsubsection{Download Protocol}
%%%%%            Call downloads to download to local machine.
%%%%%
%%%%%        \subsection{AWS}
%%%%%            \subsubsection{API'S}
%%%%%            Use AWS API that looks at file structs.
%%%%%
%%%%%            \subsubsection{Download Protocol}
%%%%%            Call download API to download to local machine.
%%%%%
%%%%%        \subsection{Authentication}
%%%%%        The Application should retain a user's login token for the various platforms to minimize the frequency they need to authenticate to the different VM platforms.
%%%%%        An option for them to log out of these sessions should also be provided.
%%%%%
%%%%%            \subsubsection{VMWare Auth (SAML)}
%%%%%            Use specific authorization based on the VMWare standard.
%%%%%
%%%%%            \subsubsection{AWS Authentication}
%%%%%            Use specific authorization based on the AWS standard.
%%%%%
%%%%%        \subsection{Web Page}
%%%%%            \subsubsection{Django}
%%%%%            Provide input box for URL: should also provide login and logout usage.
%%%%%
%%%%%        \subsection{Download Structure}
%%%%%
%%%%%        \subsection{Database?}
%%%%%

        \subsection{Web Application}
        The web server that serves the application will be Django's built in development server and when the project is completed, Cypherpath will integrate
        the web application into their existing ecosystem.

        
        \subsection{Authentication API's}
        The application will authenticate to the cloud platforms with various authentication methods and API's. To start with, the application will need to support
        SAML and AWS Authentication, but will need to be designed in a modular fashion so that other authentication methods can be added later on such as OpenID and OAuth.


        \subsection{Cloud Storage API's}
        Once the application is authenticated to a Cloud Storage platform it will gather the directory structure for showing to the user and allowing them to select
        which of their files they would like to download. To start with the application will need to support interacting with VMware and Amazon Web Services. The application will
        need to be designed so that these interactions are modular to enable adding support for other cloud platforms later. Other cloud platforms that will be included, time permitting,
        are Citrix, Google Drive and Dropbox.


        \subsection{Local Storage}
        Once the user has selected which files to download from their cloud platform, the application will download those
        files to the user's local storage. Nothing else needs to be done with the files because after the application is finished
        and Cypherpath has integrated the web application into their platform, they will direct the downloads where they need them.


    \section{Operation}
    In the following sections the operation of the application will be described, including starting the application
    (invocation), the commands the application uses and finally, how to close or terminate the application.

        \subsection{Invocation}
        Since the application will be integrated into Cypherpath's platform, the application will be started using Django's built-in development webserver and
        the user will not need to be authenticated to use the application. To invoke the application the user will simply enter the URL that the application is running on and will
        be presented with the screen of the application.

        Initially the user will be displayed with an area to enter a URL that their VM's reside at (An AWS or VMware link). If the user has previously entered and logged into one of these
        URL's before then they will also have a list of these to log back into them.

        \subsection{User Actions}
            \subsubsection{Enter URL}
            The user can enter a URL that goes to their cloud platform (VMware or AWS initially). The application will figure out which cloud platform the URL belongs to and then will have the
            user enter their login credentials for that platform. Once authenticated the application will present the user with their files and folders that are on that platform.
            
            \subsubsection{Select URL}
            In addition to the text box to enter a URL the user will also be presented with a list of previously entered URLs. The user will be able to remove URL's from this list, reconnect to them
            or close the session with that cloud platform.

            \subsubsection{Display File Structure}
            Once the user has selected a cloud platform and is authenticated, they will be presented with their directory structure on that platform. From here they can navigate through their folders and
            select multiple files and folders that they wish to download.

            \subsubsection{Download}
            Once the user has selected which files they wish to download and press the download button the application will download those files to the user's local storage. The user won't choose where to store these
            files and where the application stores them won't matter as later on the location will be chosen by Cypherpath.

            \subsubsection{Error Catching}

        \subsection{Termination}
        Since the user is not authenticated to the web application, to terminate their interaction with the application they will simply close the browser tab or window that has the
        application open. The application will keep any authentication tokens for use the next time the user uses the application.


    \section*{References}
    % Reference the project plan?
    % Possibly reference documentation to the various API's?

    \appendix
    \section{Appendix}
    % Sample files, if any?
    % Maybe a sketch of the general page layout?

\end{document}
