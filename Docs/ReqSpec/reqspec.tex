\documentclass{article}
\usepackage[margin=1in]{geometry}
\usepackage{fancyhdr}
\usepackage{graphicx}
\usepackage{vhistory}
\usepackage[parfill]{parskip}
\graphicspath{{./}}

% Set fancy looking header/footer and move page number to the right
\pagestyle{fancy}
\fancyhead{}
\fancyfoot{}
\fancyfoot[R]{\thepage}

\title{}
\author{}
\date{}

\begin{document}

    % For large document with titlepage:
    \pagenumbering{gobble}
    \begin{titlepage}
    \begin{center}
        \vspace*{1cm}

        \Huge
        \textbf{Requirements Specification}

        \vspace{.5cm}
        \LARGE
        Captain Cybeard: Neil Before Us

        \vspace{1cm}

        \textbf{Ryan Breitenfeldt \textbar\ Noah Farris\\ Trevor Surface \textbar\ Kyle Thomas}

        \vspace{.2cm}
        \Large
        September 27, 2019

        \vspace{2cm}
        \includegraphics[scale=1]{logo}

        \vfill

        Washington State University Tri-Cities\\
        CptS 421 Software Design Project 1

    \end{center}
\end{titlepage}



    \tableofcontents
    \listoffigures

    \newpage
    \begin{versionhistory}
        \vhEntry{0.2}{10.10.2019}{RB NF TS KT}{Filled in Environment \& Operation sections}
        \vhEntry{0.1}{09.27.2019}{KT}{Document Creation}
    \end{versionhistory}
    \newpage

    \pagenumbering{arabic}
    \section{Introduction}
    This application is to give the user the ability to download their virtual machines for upload to Cypherpath’s resiliency platform. It will be a Django web application in python3. The user will be able to get the virtual machines they have in the VMWare cloud and AWS cloud. It should detect what service is from the URL supplied by the user. The user will be given a selectable list of virtual machines on their desired cloud service for them to chose from. The user will select from the list and download the wanted virtual machine. The virtual machine will then be downloaded to the local computer. 
    The requirements from Cypherpath are detailed here in how Cypherpath desires to accomplish this program. This program will later be expanded by Cypherpath to be integrated into their software.     
    
    \section{Background}
    % About Cypherpath and their resiliency platform
    % Explain what this application is supposed to solve at a high, nontechnical level

    \section{Overview}
    % How the application will solve the problem, what environment it will run in, language, platform, etc
    % Overview of what it will do

    \section{Environment}
    The environment that the software will preside in consists of modules to interact with \textbf{VMWare}, 
    \textbf{AWS}, other virtual machine platforms and will also be interacting with authentication modules for those
    various platforms. The user will interact with the API's of these cloud services and authentication
    methods through a Django Web App. Lastly, the environment will consist of the application storing the selected
    virtual machines onto the user's local machine.

    %%%%%%%%%%%%%%% DIAGRAM STILL NEEDS TO BE FIXED!!! %%%%%%%%%%%%%%%%%%%%%%%%%%
    \begin{figure}[h]
    \includegraphics[scale=.7]{diagram}
        \caption{The application environment}
    \end{figure}
    %%%%%%%%%%%%%%% DIAGRAM STILL NEEDS TO BE FIXED!!! %%%%%%%%%%%%%%%%%%%%%%%%%%

        \subsection{VMWare (VSphere)}

            \subsubsection{API'S}
            Use VMWare API that looks at file structs.

%            Example URL:

            \subsubsection{Download Protocol}
            Call downloads to download to local machine.

        \subsection{AWS}
            \subsubsection{API'S}
            Use AWS API that looks at file structs.

            \subsubsection{Download Protocol}
            Call download API to download to local machine.

        \subsection{Authentication}
        The Application should retain a user's login token for the various platforms to minimize the frequency they need to authenticate to the different VM platforms.
        An option for them to log out of these sessions should also be provided.

            \subsubsection{VMWare Auth (SAML)}
            Use specific authorization based on the VMWare standard.

            \subsubsection{AWS Authentication}
            Use specific authorization based on the AWS standard.

        \subsection{Web Page}
            \subsubsection{Django}
            Provide input box for URL: should also provide login and logout usage.

        \subsection{Download Structure}

        \subsection{Database?}

    \section{Operation}
    In the following sections the operation of the application will be described, including starting the application
    (invocation), the commands the application uses and finally, how to close or terminate the application.

        \subsection{Invocation}
            \subsubsection{Web Application}

        \subsection{Commands}
            \subsubsection{Download}
            
            \subsubsection{Load File Structure}

            \subsubsection{Error Catching}

            \subsubsection{Authentication}

        \subsection{Termination}
            \subsubsection{Logout User}

            \subsubsection{Closing Application}

    \section*{References}
    % Reference the project plan?
    % Possibly reference documentation to the various API's?

    \appendix
    \section{Appendix}
    % Sample files, if any?
    % Maybe a sketch of the general page layout?

\end{document}
