\documentclass{article}
\usepackage[margin=1in]{geometry}
\usepackage{fancyhdr}
\usepackage{graphicx}
\usepackage{vhistory}
\usepackage[parfill]{parskip}
\graphicspath{{./}}

% Set fancy looking header/footer and move page number to the right
\pagestyle{fancy}
\fancyhead{}
\fancyfoot{}
\fancyfoot[R]{\thepage}

\title{}
\author{}
\date{}

\begin{document}

    % For large document with titlepage:
    \pagenumbering{gobble}
    \begin{titlepage}
    \begin{center}
        \vspace*{1cm}

        \Huge
        \textbf{Requirements Specification}

        \vspace{.5cm}
        \LARGE
        Captain Cybeard: Neil Before Us

        \vspace{1cm}

        \textbf{Ryan Breitenfeldt \textbar\ Noah Farris\\ Trevor Surface \textbar\ Kyle Thomas}

        \vspace{.2cm}
        \Large
        September 27, 2019

        \vspace{2cm}
        \includegraphics[scale=1]{logo}

        \vfill

        Washington State University Tri-Cities\\
        CptS 421 Software Design Project 1

    \end{center}
\end{titlepage}



    \tableofcontents
    \listoffigures

    \newpage
    \begin{versionhistory}
        \vhEntry{0.3}{10.15.2019}{TS}{Completed Overview}
        \vhEntry{0.2}{10.10.2019}{RB NF TS KT}{Filled in Environment \& Operation sections}
        \vhEntry{0.1}{09.27.2019}{KT}{Document Creation}
    \end{versionhistory}
    \newpage

    \pagenumbering{arabic}
    \section{Introduction}
		
    \section{Background}

    \section{Overview}
    The purpose of this application is to allow users to download files from online Virtual Machines (VM), to local file storage.
    The application itself will operate in the web browser using Python3 and Django. Upon starting the application the user would
    provide a URL of their VM. Processing the URL the user is then asked to provide credentials to validate they are the owner of
    the VM. After completing the verification, the application will then use the API of the client the user has their VM stored on
    to access their folders. The User will be shown their folder structure and chose to download the files locally to their desktop.
    Upon acceting to download locally the application will use another API to download the files local to their VM.

    The application will be written with django and python3, including the requests libraries for API calls, SAML for Authentication
    and standard libraries.

    The application only allows users to view the files from VM's they own, and download those file locally to the machine they 
    are running the application through. The users will have the ability to provide either a VMWare URL to download from, or a 
    Amazon Web Service (AWS) VM URL, to download as well. The application will be designed with modularity in mind to allow further 
    development with different VM service providers.  
    \section{Environment}

        \subsection{VMWare (VSphere)}

            \subsubsection{API'S}
            Use VMWare API that looks at file structs.

            Example URL:

            \subsubsection{Download Protocol}
            Call downloads to download to local machine.

        \subsection{AWS}
            \subsubsection{API'S}
            Use AWS API that looks at file structs.

            \subsubsection{Download Protocol}
            Call download API to download to local machine.

        \subsection{Authentication}
            \subsubsection{VMWare Auth (SAML)}
            Use specific authorization based on the VMWare standard.

            \subsubsection{AWS Auth}
            Use specific authorization based on the AWS standard.

        \subsection{Web Page}
            \subsubsection{Django}
            Provide input box for URL: should also provide login and logout usage.

        \subsection{Download Structure}

        \subsection{Database?}

    \section{Operation}
        \subsection{Invocation}
            \subsubsection{Web Application}

        \subsection{Commands}
            \subsubsection{Download}
            
            \subsubsection{Load File Structure}

            \subsubsection{Error Catching}

            \subsubsection{Authentication}

        \subsection{Termination}
            \subsubsection{Logout User}

            \subsubsection{Closing Application}

    \section*{References}

    \appendix
    \section{Appendix}

\end{document}
