\subsection{Dropbox}
This class is used in the Resiliency Platform Tool to assist users in the transfer of selected Dropbox files.
It will connect to the user interface and allow them to log into their Dropbox through a redirect. After logging
in, they will have a list view of the contents of their Dropbox. Per Dropbox's API, some files may be locked and 
not displayed. Using a check box the user will be prompted to select unique files, or to select all. Upon clicking 
the download button, the files will be uploaded to the host server to a specified location provided by the owner of
the server.

For future developers, in order to properly use this class, several changes will need to occur. An app must be created
through the Dropbox.com/developer app-console. When creating the application the developer will be given an API key and 
API secret to be used, there are other configurations that will need to be made that are explained further in this document. 

\subsubsection{Associations}
\textbf{Is contained in UI} \\

\textbf{Makes calls to Dropbox} \\

\subsubsection{Attributes}
\textbf{dbx:Object} \\
    This attribute is the main Dropbox object. Generated after the authentication, creating access to the users Dropbox. The methods contained 
    within this object will be used in the methods listed below. Which will interact with the Dropbox API to gather the users files and folders.
    Then process the download for the objects. 

\textbf{\_\_dropbox\_api\_key:String} \\
    This attribute will store the value of the API key generated by the Dropbox application development software 
    in order to interact with Dropbox's SDK provided for python. Without the key, the API calls will fail during 
    the authentication process. This attribute is private and stored within the class itself, generated by an API
    key file included within the website.

    To generate the API key, the developers must create an associated app within the dropbox.com/developer website,
    The generated API key must be stored within an API\_keys.py file within the platforms directory. The string itself
    must also be stored under the variable name Dropbox\_Api\_key, or the current attribute assignment will fail. 

\textbf{\_\_dropbox\_api\_secret:String} \\
    This attribute will store the value of the API secret that is generated by the Dropbox application development
    software. It will be used to interact with the Dropbox SDK for python, allowing easier management of the Oauth2
    interface. This attribute is private and stored within the class itself, generated by the API key file included 
    within the website.

    To generate the API secret, the developers must create an associated app within the Dropbox.com/develops website,
    The developers, must reveal the secret show within the application console of the developer website for Dropbox. When 
    used the string must be saved to the API\_keys.py file, within the platforms directory. The variable storage name must
    be Dropbox\_Api\_secret, otherwise the current functionality of the application will fail. Reference API\_keys.py for 
    examples. 

\textbf{\_\_dropbox\_authentication\_oauth\_result:Object} \\
    This attribute will store the object generated by dropbox.oauth.DropboxOAuth2Flow.finish() call. This attribute is used after validating
    the user when creating the main Dropbox interaction object. Allowing the software to interact directly to the users Dropbox storage, 
    which will be used in the below functions to manage Dropbox files, and folders for the user to select and then download. This object contains 
    an .access\_token attribute that will be passed into the dbx attribute to instantiate the Dropbox user Object.

\textbf{\_\_dropbox\_get\_files\_return:Object} \\
    This attribute stores the result from the Dropbox api method dropbox.dropbox.Dropbox.files\_list\_folder(). Storing the object 
    dropbox.files.Metadata() generated by the API call. This data will then be stored individually to allow the user to look through 
    files and select the chosen to be downloaded.

\textbf{\_\_dropbox\_get\_files\_list\_result:Object List} \\
    This attribute stores the result from the Dropbox API method dropbox.dropbox.Dropbox.files\_list\_folder().entries. Storing the metadata for 
    the various files that are gathered through the API call. The metadata will include the file name, and extension. The full path to the file, 
    which will be used when the user selects the items in which they want to download, and a parent shared id, which will not be used in this 
    implementation.

\textbf{\_\_dropbox\_download\_path:String="/Downloads"} \\
    This attribute will hold the path value used for downloading files. For this application, the download functionality will not download to the users
    main OS. The files will be downloaded directly to the server that is hosting the website, as per the requirement specification. 

\textbf{dropbox\_fromat\_dict:Object Dict Value: \{'path':'', 'dirs':[], 'files':[]\}} \\
    This attribute is a specified dict that is used for the Django Views object within the Django platform. This attribute can be interchanged
    with the operation of the view object, if the developer wanted to display the directories and subsequent included files. This is the default 
    created dict when the files are retrieved from the Dropbox user account.

\textbf{dropbox\_flat\_dict:Object Dict Value: \{'files':[]\}} \\
    This attribute is a specified dict that is used for the Django Views object within the Django platform. This attribute is the currently set 
    view default, it shows only the files available to the user, excluding the directories included within. 

\textbf{\_\_dropbox\_files\_paths:Object List} \\
    This attribute stores the paths of the files returned from the dropbox.get\_file\_list\_folder(). This attribute is used for the paths when 
    passing the necessary argument to the dropbox.files\_download\_to\_file function. 

\subsubsection{Methods}
\textbf{dropbox\_auth\_flow: In:request} \\
    This method implements the authentication pattern used by the Dropbox Python SDK. This function uses the dropbox.oauth.DropboxOAuth2Flow() 
    function to return an OAuth object. To use the command, a redirect\_uri needs to be specified. This requires the developer to specify the 
    location in the Django application where the dropbox-auth-finish/ path is located. Additionally within the dropbox.com/developer app console
    the developer needs to add the redirect URI to the console in order for the application to work properly. The final required component is a 
    'dropbox-auth-csrf-token' to be called within the object. This is used to mitigate attempts at cross site scripting.

\textbf{dropbox\_authentication\_start: In:web\_app\_session} \\
    This method redirects the user to the Dropbox authentication website. The authorization\_url is generated by the previous method returning 
    the OAuth2 object, and using the command .start() and passing in the web\_app\_session as well. After the authorize URL is generated, the 
    method will then redirect the user to the required page. 

\textbf{dropbox\_authentication\_finish: In:request} \\
    This method finalizes the authentication necessary to gain access to the user's Dropbox. Calling the dropbox\_auth\_flow method passing in
    request, and calling the .finsh() method, passing in request.GET to the method will generate the oauth result which contains the access\_token 
    needed to instantiate the Dropbox user object. The method will be wrapped in a try block in case http errors occur during login. Include bad login state
    CSRF errors from a cross site scripting attempt, Invalid credentials and a catch all for remaining errors. After the authentication has returned 
    successfully, the dropbox object is then created. After instantiating the Dropbox user object, the files will be collected from their account, 
    finally both the format method and the flatten method will be called to pass the required dict to the Views page. Finally the user will be redirected
    to the '/files' page where they will be able to select the files they wish to download.

\textbf{dropbox\_get\_files\_list:void} \\
    This method uses the dbx attribute generated after authorization of the Dropbox python SDK has completed. Using a method from the dbx object
    dbx.files\_list\_folder() with the arguments, "" to indicate root access, and recursive=True, to allow iteration through the entire Dropbox
    folder system. The call will return and object that will be stored in the dropbox\_get\_files\_listing. Which is then parsed, and placed into
    based off of the entries attribute of the object, and stored locally in dropbox\_get\_files\_list\_result. The method will then iterate through
    the entries, attribute of the object returned, and store them in a List to later be manipulated by the class.

\textbf{dropbox\_format\_entries\_recur: In: build\_dict, level, file\_list} \\
    This method uses a recursive algorithm to generate a multi leveled dict that contains the users directories, and included files, and full
    paths needed to display the information. With the current implementation the detected difference between a file and a folder is the application
    users propensity to add a '.' within a filename, and not within their directories. However if this were to occur there would be an invalid file
    type during download. This was not specified to change via the requirement specification. Via the recursive algorithm, a file\_list will be 
    generated that contains all the currently discovered files, the level specifies how deep within the folder tree the current folder and file sets 
    are located. When a folder is discovered, the method is called again passing in the new directory and finding all files associated with the folder.

\textbf{dropbox\_dict\_flatten\_recur: In:dir} \\
    This method uses a recursive algorithm to flatten the previous method dict. This is used for the current implementation of the project. If 
    a developer wishes to use the directory structure to help specify files, this function is unneeded. The recursive algorithm appends to a list the 
    files received for the various folders, when a folder is detected, the method is called again and the files are appended to the list.

\textbf{dropbox\_download\_selected\_entries: In:download\_list} \\
    This method will use the dbx.files\_download\_to\_file() and dbx.files\_download\_zip\_to\_file(). It will use the locally saved attribute 
    dropbox\_entries\_to\_download\_list to iterate through, calling the API using the developer defined dropbox\_download\_path. The function 
    will then download all the files within the list, to the location specified on the website host server. The user will be prompted with an 
    error if a file is not downloadable, or if a download failed for other reasons.

