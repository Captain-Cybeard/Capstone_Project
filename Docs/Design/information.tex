\section{Information Repositories}
The application uses several different information repositories. There will be one for each platform supported as
outlined in the \texttt{Requirements Specification} \cite{reqs}. For example, AWS storage will be an information
repository as well as Dropbox. The user's local hard drive will also be an information repository during
development and then once the client takes delivery their internal storage will become the information repository.

    \subsection{Cloud Platform Repository}
    The Cloud Repositories are the storage on the cloud for each platform that the user has files stored on.
    These files can be any type but will typically be an operating system ISO for the client's use case. These
    repositories can contain any arbitrary number of files and folders stored within this repository, the users
    will be able to navigate these through the application and select which ones to download. This information repository
    will be read only for the application and the user, they will not be deleting or adding contents to the cloud platform.
    
    \subsection{Local Storage or Internal Infrastructure}
    The second type of information repository that the application will contain is the location that the
    application will download the user selected files from. This will be the user's local storage during the
    development cycle and in production the client's internal infrastructure will be the repository for these
    files. This information repository will not be read from, it will only be written too. If the user would like to alter
    these files then they will need to use another application since the \textit{Downloader} application does not
    include the user's hard drive in it's scope.


