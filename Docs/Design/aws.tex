\subsection{AWS}
The class `aws' will be the primary handler of the application's interaction with Amazon Web Services (AWS),
this includes Amazon Elastic Compute Cloud (EC2) and Amazon Simple Storage Service (S3). The application
can use the `aws' class to get a list of all available files as well as the ability too download said files, 
given that the user has access to do both.

\subsubsection{Associations}
\textbf{Main} \\
	The AWS class is contained in `main', which will pass in authentication information and receive information about files 
	that a user has access to.

\textbf{AWS} \\
	The AWS Class makes calls to Amazon Web Services using Amazon's boto3 Python module.

\subsubsection{Attributes}
\textbf{aws\_ec2\_client\_flow:Object boto3.client()} \\
    This is an instance of the EC2 client, generated by the authentication() method using boto3.client(). Uses access\_key, access\_key\_id, and the region (by default, the region is set to 'us-west-2'). 

\textbf{aws\_s3\_client\_flow:Object boto3.client()} \\
    This is an instance of the S3 client, generated by the authentication() method using boto3.client(). Uses access\_key, access\_key\_id, and the region (by default, the region is set to 'us-west-2').

\textbf{aws\_s3\_resource\_flow:Object boto3.resource()} \\
    This is an instance of the S3 resource, generated by the authentication() method using boto3.resource() method. Uses access\_key, access\_key\_id, and the region (by default, the region is set to 'us-west-2').

\textbf{aws\_get\_files\_list\_result:Object JSON} \\
    This attribute stores all files returned by the AWS API that a user can see.

\textbf{aws\_entries\_to\_download\_list:Object List} \\
    This attribute will store the information provided by the user of which files they want to download. The object is the same as the above 
    attributes. Using the entries.path\_lower, attribute of the generated object, the information will be provided to the download method in 
    order to download the list of files specified by the user.

\textbf{aws\_download\_path:String=``C:/Downloads''} \\
    This attribute will hold the path value used for downloading files. The default for this attribute will be the downloads folder. The user 
    will be given the option to specify the download location. This will update the value to their new destination. If the destination does 
    not exist, the user will be prompted.

\subsubsection{Methods}
\textbf{aws\_authentication:void} \\
    This method passes a AWS access key and AWS access key\_id to Amazons boto3 module, which is used to interact with AWS through python. The key 
    and keyID will be used to create three class attributes that will be used throughout the aws class. These are s3\_client, ec2\_client, s3\_resource.
    
\textbf{aws\_get\_files\_list:void} \\
    Using the \_ec2\_client defined in the authentication method, the `get\_files\_list' method will be used to retrieve all available files that the user 
    has access to. This will use the client.describe\_images method as defined in the boto3 specifications.

\textbf{aws\_format\_entries\_list:void} \\
    This method will format the returned JSON from get\_files\_list according to the specifications laid out for JSON formatting. As is specified in the
    Dropbox section of this document: ``the ordering will be a hierarchal order, listing folders and the included subfolders and files. This will then be stored in json format to be used by the website to display the data in an
    accordion list which will expand the structures to display subfolders and files included within selected folder. Providing an easier to manage
    interface for the user.''

\textbf{aws\_select\_entries\_to\_download:void} \\
    This method will take input from the UI and will generate a key value list of items to download through the S3 Client. This list will be saved locally to this class,
    and will be used by the 'download\_selected\_entries' method.

\textbf{aws\_download\_selected\_entries:void} \\
    This method will use the \_s3\_resource.download\_file() method within a for loop to download all files specified in the variable `entries\_to\_download\'.

% percent signs are comments
% \textbf{} makes the stuff in {} bold
% \textit{} is italics
% \\ makes a new line
% space between lines makes a new paragraph
% please escape underscores and & with a backslash
