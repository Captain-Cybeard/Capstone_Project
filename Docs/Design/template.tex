\subsection{class class\_template}
This class exists to allow future developers to add more cloud platforms to the product. Providing both convention for use, and layout 
for attribute creation and method creation. Included are basic methods that simplify the authentication. A Method to help format the 
API return data to comply to the standard specified. A Method to simplify the UI return to parse and complete the download for the user.
Other methods are used for helping simplify the class structure for future development. 

\subsubsection{Associations}
The association \textbf{UI Interface} is used when developing a new cloud platform class to be added into the application. This will also 
include any necessary inputs for the developer to adjust inorder for the class to communicate with the UI. 

\subsubsection{Attributes}
\textbf{[NAME]\_?[METHOD]\_[PARAM]:[TYPE]} \\
This attribute is describes the convention for future developers to use when describing attributes to add to new classes. 
If an attribute is not inherent of a method or [METHOD] as described in the attribute, it is passed in with [NAME]\_\_[PARAM].
Helping to provide readability to future developers, and indicating how the attributes are being used through the application.

\textbf{[NAME]\_\_user\_download\_path:String} \\
This attribute needs to be included in every class, and will default to the Users C:/Downloads, unless specifically alterd by the user 
through the application itself. 

/textbf{[NAME]\_get\_files\_list\_result:List} \\
This attribute is a recomendation to include in every class as a storage location for the return from the cloud api. This will be 
utilized in methods included within the template. 

/textbg{[NAME]\_entries\_to\_download\_list:List} \\
This attribute is a recomendation to include in every class as storage for the return value of the UI for the cloud service. By using
this attribute, the developer can parse the return in the fasion to which they need to allow downloading the user specified data.

\subsubsection{Methods}
\textbf{[NAME]\_authentication(self):type} \\
This method is used to simplify api usage for authentication and will interact with the UI in a more seemless manner. Additionall authentication
methods that need to be created should include [NAME]\_authentication\_[FUNCTION] to provide clear connection to the developer and future 
developers. This method should only process any necessary procedures when authenticating the user to the cloud service the class is designed around

\textbf{[NAME]\_get\_files\_list(self):type} \\
This method is used to simplify the api usage for gathering the neccessary metadata acossiated with cloud platform the class is being created for.
The return data should be saved local to this class in order to be used by future classes. 

\textbf{[NAME]\_format\_entries\_list(self):type} \\
This method is used to format the returned data provided by the [NAME]\_get\_files\_list(self) method. The format includes a JSON defined as 
\begin{verbatim}
\{
   "path":"",
   "directories":[
      \{
         "path":"",
         "directories":[],
         "files":[]
      \}
   ],
   "files":[
      \{
         "path":"",
         "name":""
      \}
   ]
\}
\end{verbatim}
This JSON will be passed to the UI for the user to interact with when selecting what they want to download.

\textbf{[NAME]\_select\_entries\_to\_download(self):type} \\

\textbf{[NAME]\_get\_user\_download\_path(self):type} \\
This method will change the donwload path provided by the user through the UI. The UI will pass the string into this function to change the
local attibute acossiated with it.

\textbf{[NAME]\_download\_selected\_entries(self, list\_of\_selected\_downloads=null, [NAME]\_path):type} \\
This method is used to simplify the api usage for downloading. Once a user has selected the items they wish to download, and decided upon the 
download path. This function will use the cloud platform specified by the class to then download the list of items returned by the UI, into the 
specified user folder.

