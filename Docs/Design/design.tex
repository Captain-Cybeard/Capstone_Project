% Compiled with pdflatex
\documentclass{article}
\usepackage[margin=1in]{geometry}
\usepackage{fancyhdr}
\usepackage{graphicx}
\usepackage{vhistory}
\usepackage[parfill]{parskip}
\usepackage{float}
\graphicspath{{../Images/}}

\title{}
\author{}
\date{}

\fancyfoot{}
\fancyhead{}
\fancyfoot[R]{\thepage}

\begin{document}
\pagenumbering{gobble}

\begin{titlepage}
    \begin{center}
        \vspace*{1cm}

        \Huge
        \textbf{Requirements Specification}

        \vspace{.5cm}
        \LARGE
        Captain Cybeard: Neil Before Us

        \vspace{1cm}

        \textbf{Ryan Breitenfeldt \textbar\ Noah Farris\\ Trevor Surface \textbar\ Kyle Thomas}

        \vspace{.2cm}
        \Large
        September 27, 2019

        \vspace{2cm}
        \includegraphics[scale=1]{logo}

        \vfill

        Washington State University Tri-Cities\\
        CptS 421 Software Design Project 1

    \end{center}
\end{titlepage}



\tableofcontents
\newpage
\listoffigures
\newpage

\begin{versionhistory}
	\vhEntry{2.0}{04.11.2020}{KT}{Revisions from review session}
	\vhEntry{1.0}{03.06.2020}{RB NF TS KT}{First Draft, revisions}
	\vhEntry{0.5}{03.04.2020}{RB}{Google Section}
	\vhEntry{0.4}{03.04.2020}{NF}{AWS Section}
	\vhEntry{0.3}{03.04.2020}{TS}{Dropbox Section}
	\vhEntry{0.2}{03.04.2020}{KT}{Filled in sections other than data dictionary}
	\vhEntry{0.1}{02.25.2020}{KT}{Document Creation}
\end{versionhistory}
\newpage

\pagenumbering{arabic}
\section{Introduction}
This document will go over the design and architecture for the \textit{Downloader} application that was laid out in
\texttt{Requirements Specification}\cite{reqs}. The design document will help the client (Cypherpath) and the
software development team (Captain CyBeard) understand the architecture and functionality of the application.

Subsequent sections will go over the general architecture of the application in unified modeling language
(UML) followed by a data dictionary. Afterwards the user interface and examples of information repositories needed
by the application will be presented.

\section{Architecture}
The application will consist of ten classes in a model, view, controller architecture. The \textbf{Main} class will be
the controller for the models (the classes that wrap up the cloud platforms) and the view classes will be the three
classes that control what the user sees. The \textit{model} classes (\textbf{AWS}, \textbf{Dropbox}, and \textbf{Google})
are templates of the \textbf{Template} class. This isn't an actual class that is implemented, however it is the format used
when creating a cloud platform model and is provided for future maintainers of the project to more easily add classes.

The following UML diagrams will show the classes and their relationships with
each other, along with their attributes and methods.

%%
% UML diagrams go here
%%

\begin{figure}[H]
\includegraphics[scale=.25]{des_all}
\centering
\caption{The overall application architecture.}
\end{figure}

\begin{figure}[p]
\includegraphics[scale=.5]{des_template}
\centering
\caption{The template class all platforms follow}
\end{figure}

\begin{figure}[p]
\includegraphics[scale=.5]{des_aws}
\centering
\caption{The AWS Class.}
\end{figure}

\begin{figure}[p]
\includegraphics[scale=.5]{des_dropbox}
\centering
\caption{The Dropbox Class.}
\end{figure}

\begin{figure}[p]
\includegraphics[scale=.5]{des_google}
\centering
\caption{The GDriveDownloader Class.}
\end{figure}

\begin{figure}[p]
\includegraphics[scale=.5]{des_views}
\centering
\caption{The Controller and Views Classes.}
\end{figure}

\begin{figure}[p]
\includegraphics[scale=.5]{des_json}
\centering
\caption{The JSON Classes.}
\end{figure}

%%
% Data dictionary section, keep main at the top since it includes the proper section heading
%%

\section{Data Dictionary}
The Data Dictionary sections will describe what each class does, along with a more detailed view of its
associations, attributes and methods that the class has.

\subsection{Main}

\subsubsection{Associations}
\textbf{AWS} \\
The 'Main' class contains the AWS class.

\textbf{Dropbox} \\
The 'Main' class contains the Dropbox class.

\textbf{GDriveDownloader} \\
The 'Main' class contains the GDriveDownloader class.

\textbf{File\_Selection} \\
The 'Main' sends Directorys back and forth to the File\_Selection class.

\textbf{Platform\_Selection} \\
The 'Main' class sends platforms back and forth to the Platform\_Selection class.

\textbf{Authentication} \\
The 'Main' class sends field names back and forth to the Authentication class.

\subsubsection{Attributes}
\textbf{} \\

\textbf{} \\

\subsubsection{Methods}
\textbf{} \\
  
\textbf{} \\


\subsection{Directory}
The \textbf{Directory} class is a JSON friendly class and will be used to display the directory
structure and files to the user.

\subsubsection{Associations}
\textbf{} \\

\textbf{} \\

\subsubsection{Attributes}
\textbf{name: String} \\
The attribute \textbf{name} is a string that contains the name of the directory.

\textbf{path: String} \\
The attribute \textbf{path} is a string that contains the path of the directory
relative to the user's home directory.

\textbf{directories: Directory Vector} \\
The \textbf{directories} attribute is a list of \textbf{Directory} objects that
are contained in the directory.

\textbf{files: File Vector} \\
The attribute \textbf{files} is a list of \textbf{File} objects that are contained
in the directory.

\subsubsection{Methods}
The \textbf{Directory} class does not contain any methods.


\subsection{File}
The class \textbf{File} is a JSON friendly class and will be used to display the files to the user.

\subsubsection{Associations}
\textbf{} \\

\textbf{} \\

\subsubsection{Attributes}
\textbf{name: String} \\
The \textbf{name} attribute is a string that contains the name of the file.

\textbf{path: String} \\
The \textbf{path} attribute is the path the file is in relative the the user's
root directory.

\subsubsection{Methods}
The \textbf{File} class does not contain any methods.


\subsection{class class\_template}
This class exists to allow future developers to add more cloud platforms to the product. Providing both convention for use, and layout 
for attribute creation and method creation. Included are basic methods that simplify the authentication. A Method to help format the 
API return data to comply to the standard specified. A Method to simplify the UI return to parse and complete the download for the user.
Other methods are used for helping simplify the class structure for future development. 

\subsubsection{Associations}
The association \textbf{UI Interface} is used when developing a new cloud platform class to be added into the application. This will also 
include any necessary inputs for the developer to adjust inorder for the class to communicate with the UI. 

\subsubsection{Attributes}
\textbf{[NAME]\_?[METHOD]\_[PARAM]:[TYPE]} \\
This attribute is describes the convention for future developers to use when describing attributes to add to new classes. 
If an attribute is not inherent of a method or [METHOD] as described in the attribute, it is passed in with [NAME]\_\_[PARAM].
Helping to provide readability to future developers, and indicating how the attributes are being used through the application.

\textbf{[NAME]\_\_user\_download\_path:String} \\
This attribute needs to be included in every class, and will default to the Users C:/Downloads, unless specifically alterd by the user 
through the application itself. 

/textbf{[NAME]\_get\_files\_list\_result:List} \\
This attribute is a recomendation to include in every class as a storage location for the return from the cloud api. This will be 
utilized in methods included within the template. 

/textbg{[NAME]\_entries\_to\_download\_list:List} \\
This attribute is a recomendation to include in every class as storage for the return value of the UI for the cloud service. By using
this attribute, the developer can parse the return in the fasion to which they need to allow downloading the user specified data.

\subsubsection{Methods}
\textbf{[NAME]\_authentication(self):type} \\
This method is used to simplify api usage for authentication and will interact with the UI in a more seemless manner. Additionall authentication
methods that need to be created should include [NAME]\_authentication\_[FUNCTION] to provide clear connection to the developer and future 
developers. This method should only process any necessary procedures when authenticating the user to the cloud service the class is designed around

\textbf{[NAME]\_get\_files\_list(self):type} \\
This method is used to simplify the api usage for gathering the neccessary metadata acossiated with cloud platform the class is being created for.
The return data should be saved local to this class in order to be used by future classes. 

\textbf{[NAME]\_format\_entries\_list(self):type} \\
This method is used to format the returned data provided by the [NAME]\_get\_files\_list(self) method. The format includes a JSON defined as 
\begin{verbatim}
\{
   "path":"",
   "directories":[
      \{
         "path":"",
         "directories":[],
         "files":[]
      \}
   ],
   "files":[
      \{
         "path":"",
         "name":""
      \}
   ]
\}
\end{verbatim}
This JSON will be passed to the UI for the user to interact with when selecting what they want to download.

\textbf{[NAME]\_select\_entries\_to\_download(self):type} \\

\textbf{[NAME]\_get\_user\_download\_path(self):type} \\
This method will change the donwload path provided by the user through the UI. The UI will pass the string into this function to change the
local attibute acossiated with it.

\textbf{[NAME]\_download\_selected\_entries(self, list\_of\_selected\_downloads=null, [NAME]\_path):type} \\
This method is used to simplify the api usage for downloading. Once a user has selected the items they wish to download, and decided upon the 
download path. This function will use the cloud platform specified by the class to then download the list of items returned by the UI, into the 
specified user folder.


\subsection{Dropbox}
This class is used in the Resiliency Platform Tool to assist users in the download of selected Dropbox files.
It will connect to the user interface and allow them to log into their Dropbox through a redirect. After logging
in, they will have a list view of the contents of their Dropbox. Per Dropbox's api, some files may be locked and 
not displayed. Using a check box the user will be prompted to select unique files, or to select all. Upon clicking 
the download button, the files will be sent to the C:/Downloads folder, unless specified by the user to change.

\subsubsection{Associations}
\textbf{Is contained in UI} \\

\textbf{Makes calls to Dropbox} \\

\subsubsection{Attributes}
\textbf{dropbox\_api\_key:String} \\
    This attribute will store the value of the Api key generated by the Dropbox application development software 
    in order to interact with dropbox's SDK provided for python. Without the key, the Api calls will fail during 
    the authentication process. This attribute is private and stored within the class itself, generated by an Api
    key file included within the website.

\textbf{dropbox\_api\_secret:String} \\
    This attribute will store the value of the Api secret that is generated by the Dropbox application development
    software. It will be used to interact with the Dropbox SDK for python, allowing easier management of the Oauth2
    interface. This attribute is private and stored within the class itself, generated by the Api key file included 
    within the website.

\textbf{dropbox\_authentication\_auth\_flow:Object} \\
    This attribute is an object that is created by the Dropbox SDK, the formal definition of this object is 
    dropbox.oauth.DropboxOAuth2FlowNoRedirect(). This object will be generated, with parameters of both the Api key
    and the Api secret. Once the object is stored local to the class, the Oauth2 flow can start. For now, the NoRedirect
    function is being called. When implemented into the website itself, this will be changed to Redirect, allowing the 
    user to log into their Dropbox within the browser. 

\textbf{dropbox\_authentication\_authorize\_url:String} \\
    This attribute stores the url that is generated by the dropbox\_authentication\_auth\_flow Object, allowing the user
    to be redirected to the Dropbox log-in. The result of which will generate a key value to be passed back into the application
    to complete the validation of the user. Additionally, when the class is integrated into the full website, this will be replaced
    with the redirect url to allow the same operation for validating the user.

\textbf{dropbox\_authentication\_auth\_code:String} \\
    This attribute stores the returned authorization code generated from the dropbox\_authentication\_authorize\_url link. When
    prompted the user will enter the string returned by visiting the link. This will be stored locally to allow the class to use the
    string in the finish() method of the Auth Flow Object. 

\textbf{dropbox\_authentication\_oauth\_result:Object} \\
    This attribute will store the object generated by dropbox\_authentication\_auth\_flow.finish() call. This attribute is used upon validating
    the user when creating the main Dropbox interaction object. Allowing the software to interact directly to the users Dropbox storage, 
    which will be used in the below functions to manage Dropbox files, and folders for the user to select and then download. 

\textbf{dbx:Object} \\
    This attribute is the main Dropbox object. Generated after the authentication, creating access to the users Dropbox. The methods contained 
    within this object will be used in the methods listed below. Which will interact with the Dropbox Api to gather the users files and folders.
    Then process the download for the objects. 

\textbf{dropbox\_get\_files\_return:Object} \\
    This attribute stores the result from the Dropbox api method dropbox.dropbox.Dropbox.files\_list\_folder(). Storing the object 
    dropbox.files.Metadata() generated by the api call. This data will then be stored individually to allow the user to look through 
    files and select the chosen to be downloaded.

\textbf{dropbox\_get\_files\_list\_result:Object List} \\
    This attribute stores the result from the Dropbox api method dropbox.dropbox.Dropbox.files\_list\_folder().entries. Storing the metadata for 
    the various files that are gathered through the Api call. The metadata will include the file name, and extension. The full path to the file, 
    which will be used when the user selects the items in which they want to download, and a parent shared id, which will not be used in this 
    implementation.

\textbf{dropbox\_entries\_to\_download\_list:Object List} \\
    This attribute will store the information provided by the user of which files they want to download. The object is the same as the above 
    attributes. Using the entries.path\_lower, attribute of the generated object, the information will be provided to the download method in 
    order to download the list of files specified by the user.

\textbf{dropbox\_download\_path:String="C:/Downloads"} \\
    This attribute will hold the path value used for downloading files. The default for this attribute will be the downloads folder. The user 
    will be given the option to specify the download location. This will update the value to their new destination. If the destination does 
    not exist, the user will be prompted.

\subsubsection{Methods}
\textbf{dropbox\_authentication:void} \\
    This method implements the authentication pattern user by the Dropbox Python SDK. This method will generate an authorization\_flow object
    that will then generate a url link when instantiated. The user will then provide credentials for the link and copy to authorization\_key 
    to be passed into the program. When integrated into the website, this will cause a redirect to a Dropbox login then load the key without 
    user interaction. After a valid key is provided, given no error is displayed to the user, the key will be stored in a dbx attribute, detailed 
    above.  

\textbf{dropbox\_get\_files\_list:void} \\
    This method uses the dbx attribute generated after authorization of the Dropbox python SDK has completed. Using a method from the dbx object
    dbx.files\_list\_folder() with the arguments, "" to indicate root access, and recursive=True, to allow iteration through the entire Dropbox
    folder system. The call will return and object that will be stored in the dropbox\_get\_files\_listing. Which is then parsed, and placed into
    based off of the entries attribute of the object, and stored locally in dropbox\_get\_files\_list\_result. The method will then iterate through
    the entries, attribute of the object returned, and store them in a List to later be manipulated by the class.

\textbf{dropbox\_format\_entries\_list:void} \\
    This function uses the local dropbox\_get\_files\_list\_result to sort the data and reorganize it. The ordering will be a hierarchal order, listing 
    folders and the included subfolders and files. This will then be stored in JSON format to be used by the website to display the data in an
    accordion list which will expand the structures to display subfolders and files included within selected folder. Providing an easier to manage
    interface for the user.

\textbf{dropbox\_select\_entries\_to\_download:void} \\
    This method will be used to store the indices selected by the user for the files they wish to download. Using key values generated through 
    the UI to then create a new list Dropbox\_entries\_to\_download\_list in which the selections will be saved local to the class.

\textbf{dropbox\_download\_selected\_entries:void} \\
    This method will use the dbx.files\_download\_to\_file(). It will use the locally saved attribute dropbox\_entries\_to\_download\_list to
    iterate through, calling the Api using the user defined dropbox\_download\_path. The function will then download all the files within the 
    list, to the location specified. The user will be prompted with an error if a file is not downloadable, or if a download failed for other 
    reasons.

% percent signs are comments
% \textbf{} makes the stuff in {} bold
% \textit{} is italics
% \\ makes a new line
% space between lines makes a new paragraph
% please escape underscores and & with a backslash

\subsection{AWS}
The class `aws' will be the primary handler of the application's interaction with Amazon Web Services (AWS),
this includes Amazon Elastic Compute Cloud (EC2) and Amazon Simple Storage Service (S3). The application
can use the `aws' class to get a list of all available files as well as the ability too download said files, 
given that the user has access to do both.

\subsubsection{Associations}
\textbf{Main} \\
	The AWS class is contained in `Files' view, which will pass in authentication information and receive information about files 
	that a user has access to as well as give them the ability to download files.

\textbf{AWS} \\
	The AWS Class makes calls to Amazon Web Services using Amazon's boto3 Python module.

\subsubsection{Attributes}
\textbf{aws\_ec2\_client\_flow:Object boto3.client()} \\
    This is an instance of the EC2 client, generated by the authentication() method using boto3.client(). Uses access\_key, access\_key\_id, and the region (by default, the region is set to 'us-west-2'). 

\textbf{aws\_s3\_client\_flow:Object boto3.client()} \\
    This is an instance of the S3 client, generated by the authentication() method using boto3.client(). Uses access\_key, access\_key\_id, and the region (by default, the region is set to 'us-west-2').

\textbf{aws\_s3\_resource\_flow:Object boto3.resource()} \\
    This is an instance of the S3 resource, generated by the authentication() method using boto3.resource() method. Uses access\_key, access\_key\_id, and the region (by default, the region is set to 'us-west-2').

\textbf{aws\_download\_path:String=``C:/Downloads''} \\
    This attribute will hold the path value used for downloading files. The default for this attribute will be the downloads folder. The user 
    will be given the option to specify the download location. This will update the value to their new destination. If the destination does 
    not exist, the user will be prompted.

\subsubsection{Methods}
\textbf{aws\_init} \\
    This method passes a AWS access key and AWS access key\_id to Amazons boto3 module, which is used to interact with AWS through python. The key 
    and keyID will be used to create three class attributes that will be used throughout the aws class. These are s3\_client, ec2\_client, s3\_resource.
    
\textbf{aws\_get\_image\_list:json} \\
    Using the \_ec2\_client defined in the \_\_init\_\_ method, the `get\_image\_list' method will be used to retrieve all available files that the user 
    has access to. This will use the client.describe\_images method as defined in the boto3 specifications. This method is used by the Files view.

\textbf{aws\_get\_buckets:json} \\
    Using the \_s3\_client defined in the \_\_init\_\_  method, the `get\_buckets' method will be used to retrieve all available buckets that the user 
    has access to.

\textbf{aws\_list\_images\_in\_bucket:json} \\
    Uses \_s3\_resource to list all images contained within a bucket, this method will return a JSON dict.

\textbf{aws\_export\_to\_bucket} \\
    This method is optionally used and gives a developer the ability to export images to a bucket.

\textbf{aws\_download\_images} \\
    This method will use the \_s3\_resource.download\_file() method within a for loop specified within the Files view to download all selected images.



\subsection{GDriveDownloader}
The GDriveDownloader class is responcable fot authenticating to Google, aquire a list of files from Google Drive, and download the file.
This class is able to take the authentication token and the built service as an input on creation.  
​
\subsubsection{Associations}
The association \textbf{UI} is used to allow the user to authenticate to their Google account and to display the avalible files.
​
​
\subsubsection{Attributes}
\textbf{GDriveDownloader_creds} \\
The attribute GDriveDownloader_creds holds the authentication token.
​
\textbf{GDriveDownloader_SCOPES} \\
The attribute GDriveDownloader_SCOPES holds the scope of what the application can access from Google.

\textbf{GDriveDownloader_service} \\
The attribute GDriveDownloader_service holds the built service.

\textbf{GDriveDownloader_file_List} \\
The attribute GDriveDownloader_file_List hold the list of files from Google Drive.

\textbf{GDriveDownloader_json} \\
The attribute GDriveDownloader_json holds the json need by the UI. 
​
\subsubsection{Methods}
\textbf{GDriveDownloader_authentication} \\
This method handles the authentication to google and gets the authentication token. It saves the token in GDriveDownloader_creds.

\textbf{GDriveDownloader_save_Token} \\
This method saves the token to a file.

\textbf{GDriveDownloader_load_Token} \\
This method loads the token from a file.

\textbf{GDriveDownloader_build_Service} \\
This method builds the service from the scope and the authentication token.

\textbf{GDriveDownloader__download_File} \\
This method downloads a file with the selected file ids. 

\textbf{GDriveDownloader__get_Files} \\
This method qurries Google Drive for the files stored in the user's Google account. It will also create the json for the UI and store it in GDriveDownloader_json.   
\subsection{Platform\_Selection}
The \textbf{Platform\_Selection} class is a view in the MVC design pattern that will present
the user with a drop-down menu that contains the available platforms for downloading.

\subsubsection{Associations}
\textbf{} \\

\textbf{} \\

\subsubsection{Attributes}
\textbf{platforms: String Vector} \\
The \textbf{platforms} attribute is a list of the strings containing the cloud platforms that
are supported by the application for the user to download.

\subsubsection{Methods}
\textbf{submit(platform: string)} \\
The method \textbf{submit} will pass back the \textbf{Main} class (the controller) a string
containing the platform that the user selected. The \textbf{Main} class will use this information
for the next view classes.
  

\subsection{Authentication}
The \textbf{Authentication} class is a view in the MVC design pattern that will retrieve the user's
credentials for a cloud platform.

\subsubsection{Associations}
\textbf{} \\

\textbf{} \\

\subsubsection{Attributes}
\textbf{fields: String Vector} \\
The \textbf{fields} attribute is a list of field names that the application will create and prompt the user
for. For example, there could be a \textit{user name} field and a \textit{password} field or a \textit{
client\_secret} token or any other variable number of fields needed that the \textbf{Authentication} class
will need to present to the user.

\subsubsection{Methods}
\textbf{authenticate(user\_input: String Vector): Boolean} \\
The method \textbf{authenticate} takes in the filled in fields from the user and sends those back to the
\textbf{Main} class. The \textbf{Main} class will reply with \textit{True} if the authentication was a
success, allowing the application to move onto the next view. If the credentials did not work for
authentication then \textit{False} will be returned and the \textbf{Authenticate} view will re-prompt the user
for their credentials.
  

\subsection{File\_Selection}
The class \textbf{File\_Selection} is a view in the MVC design pattern that will display
the available files to download to the user and allow the user to select the files.

\subsubsection{Associations}
\textbf{} \\

\textbf{} \\

\subsubsection{Attributes}
\textbf{available\_files: Directory} \\
The attribute \textbf{available\_files} contains the files that on the cloud platform to
download.

\subsubsection{Methods}
\textbf{download(dirs: Directory)} \\
The \textbf{download} method will send a message back to the \textbf{Main} class
(the controller) containing a \textbf{Directory} class with just the files that the user
has selected to download.

\section{User Interface}
The User interface section will over the actions that a user will be able to take, explaining all menus, buttons,
pull down menus, selections, etc. This section will also describe actions that system administrators will need to
take to start and stop the application.

\begin{figure}[H]
  \includegraphics[scale=.5]{user_flow}
  \centering
  \caption{The workflow for the user}
\end{figure}

    \subsection{User Interaction}
    The user will interact with the application through the web interface. They will have several screens (views)
    to navigate through. Those will include the main screen to select the platform to download from (AWS, Dropbox,
    Google Drive, etc) and then will be presented with a view to enter their credentials. Once their credentials
    are entered they will be presented with the final view which contains their directory structure and contents
    and they will be able to multi-select which of those contents they would like to download.

    For more information, readers can refer to the \texttt{Prototype Screenshots for the Downloader Application}\cite{prot}
    which will cover the work flow for the user and screenshots of the preliminary screens.
    
      \subsubsection{Platform Selection Screen}
      The first page of the application that the user will come across is a page to select which platform they will be
      viewing and downloading files from. This page will contain a \textit{pull down} menu containing each of the
      cloud platforms that the application supports. Once the user has selected the one they want, they will press the
      \textit{submit} button which will bring them to the next page.

      \begin{figure}[H]
        \includegraphics[scale=.5]{main_screen}
        \centering
        \caption{The first screen of the application.}
      \end{figure}

      \subsubsection{Authentication Screen}
      After the user has selected their platform from the \textit{platform selection screen}, the application will
      determine what kind of authorization information it needs from the user and prompt the user to enter this information.
      This will include a user name, and either a password, access token or both. Once the user enters their credentials for
      their cloud account they will press the \textit{ok} button to allow the application to authenticate to the cloud platform
      and read their files. They will be redirected to the final screen at this point.
    
      \subsubsection{File Selection Screen}
      On the final screen of the application, after the user has selected and authenticated to a cloud platform they will be
      presented with the files and directory structure that they have on that platform. The user will be able to navigate their
      directory structure from this screen as well as multi-select which files and directories they would like to download.

      Once the user is ready to download their files and directories they will click on the \textit{download} button which will
      initiate the downloading of their files to where the user's web browser directs downloads to. Once the download is complete
      the user can exit the browser window, select the \textit{start over} button to go back to the \textit{Platform Selection Screen}
      or select a different set of files to download.

      \begin{figure}[H]
        \includegraphics[scale=.5]{download_screen}
        \centering
        \caption{The screen to view and download files.}
      \end{figure}
    
    \subsection{Administrator Interaction}
    The system administrator will interact with the application by using the built-in Django commands on the
    server that the application will run on. They will mainly only be starting and stopping the web application
    through this interface. Django does have a built-in web administration page but this application will not
    need to take advantage of this functionality.

      \subsubsection{Start Application}
      To start the application the system administrator will use the Django created \textit{manage.py} file
      to invoke the application on its default port of \textit{8000}.
      \begin{verbatim}
      $ python manage.py runserver
      \end{verbatim}

      If the administrator would like to expose the webserver beyond \textit{localhost} and let other devices connect or
      use a different port then they will pass an argument with the sources and new port. For example, to let any source
      connect to the server from port \textit{8080} the administrator would type:
      \begin{verbatim}
      $ python manage.py runserver 0:8080
      \end{verbatim}
      The ``0'' before the \textit{colon} is shorthand for \textit{0.0.0.0} which will allow any source to connect.

      \subsubsection{Stop Application}
      Since the web server will be a foreground process in the terminal, the web administrator will press \textit{ctrl + c}
      to kill the process or any of the other ten thousand ways to terminate a process on a *nix system.



\section{Information Repositories}
The application uses several different information repositories. There will be one for each platform supported as
outlined in the \texttt{Requirements Specification} \cite{reqs}. For example, AWS storage will be an information
repository as well as Dropbox. The user's local hard drive will also be an information repository during
development and then once the client takes delivery their internal storage will become the information repository.

    \subsection{Cloud Platform Repository}
    The Cloud Repositories are the storage on the cloud for each platform that the user has files stored on.
    These files can be any type but will typically be an operating system ISO for the client's use case. These
    repositories can contain any arbitrary number of files and folders stored within this repository, the users
    will be able to navigate these through the application and select which ones to download. This information repository
    will be read only for the application and the user, they will not be deleting or adding contents to the cloud platform.
    
    \subsection{Local Storage or Internal Infrastructure}
    The second type of information repository that the application will contain is the location that the
    application will download the user selected files from. This will be the user's local storage during the
    development cycle and in production the client's internal infrastructure will be the repository for these
    files. This information repository will not be read from, it will only be written too. If the user would like to alter
    these files then they will need to use another application since the \textit{Downloader} application does not
    include the user's hard drive in it's scope.




\newpage
\pagenumbering{gobble}
    \begin{thebibliography}{7}
    \bibitem{pp}
    Ryan Breitenfeldt, Noah Farris, Trevor Surface, Kyle Thomas.
    \textit{Project Plan}.
    [\textit{Project Plan for Downloader Application}] 2019.

    \bibitem{reqs}
    Ryan Breitenfeldt, Noah Farris, Trevor Surface, Kyle Thomas.
    \textit{Requirements Specification}.
    [\textit{Requirements Specification for Downloader Application}] 2019.

    \bibitem{prot}
    Ryan Breitenfeldt, Noah Farris, Trevor Surface, Kyle Thomas.
    \textit{Prototype}.
    [\textit{Prototype Screenshots for Downloader Application}] 2019.


    \bibitem{cypherpath}
    Cypherpath.com. (2019). Cypherpath, Inc. [online] Available at: https://www.cypherpath.com/ [Accessed 21 Oct. 2019].

    \bibitem{aws}
    Amazon Web Services, Inc. (2019). Amazon Web Services (AWS) - Cloud Computing Services. [online] Available at: https://aws.amazon.com/ [Accessed 10 Oct. 2019].

%    \bibitem{vmware}
%    VMware. (2019). VMware – Cloud, Mobility, Networking \& Security Solutions. [online] Available at: https://www.vmware.com/ [Accessed 15 Nov. 2019].

    \bibitem{python}
    Python.org. (2019). Welcome to Python.org. [online] Available at: https://www.python.org/ [Accessed 6 Nov. 2019].

    \bibitem{django}
    Djangoproject.com. (2019). The Web framework for perfectionists with deadlines \textbar\ Django. [online] Available at: https://www.djangoproject.com/ [Accessed 6 Nov. 2019].
    \end{thebibliography}

%\appendix
%\section{Appendix}

\end{document}
