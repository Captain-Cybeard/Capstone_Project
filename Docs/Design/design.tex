% Compiled with pdflatex
\documentclass{article}
\usepackage[margin=1in]{geometry}
\usepackage{fancyhdr}
\usepackage{graphicx}
\usepackage{vhistory}
\usepackage[parfill]{parskip}
\usepackage{float}
\graphicspath{{../Images/}}

\title{}
\author{}
\date{}

\fancyfoot{}
\fancyhead{}
\fancyfoot[R]{\thepage}

\begin{document}
\pagenumbering{gobble}

\begin{titlepage}
    \begin{center}
        \vspace*{1cm}

        \Huge
        \textbf{Requirements Specification}

        \vspace{.5cm}
        \LARGE
        Captain Cybeard: Neil Before Us

        \vspace{1cm}

        \textbf{Ryan Breitenfeldt \textbar\ Noah Farris\\ Trevor Surface \textbar\ Kyle Thomas}

        \vspace{.2cm}
        \Large
        September 27, 2019

        \vspace{2cm}
        \includegraphics[scale=1]{logo}

        \vfill

        Washington State University Tri-Cities\\
        CptS 421 Software Design Project 1

    \end{center}
\end{titlepage}



\tableofcontents
\newpage
\listoffigures
\newpage

\begin{versionhistory}
	\vhEntry{0.1}{02.25.2020}{KT}{Document Creation}
\end{versionhistory}
\newpage

\pagenumbering{arabic}
\section{Introduction}
This document will go over the design and architecture for the \textit{Downloader} application that was laid out in
\texttt{Requirements Specification}\cite{reqs}. The design document will help the client (Cypherpath) and the
software development team (Captain CyBeard) understand the architecture and functionality of the application.

Subsequent sections will go over the general architecture of the application in unified modeling language
(UML) followed by a data dictionary. Afterwards the user interface and examples of information repositories needed
by the application will be presented.

\section{Architecture}
The application will consist of \textbf{\textit{NUMBER OF CLASSES}} in a model, view, controller architecture. The following
UML diagrams will show the classes and their relationships with each other, along with their attributes and methods.

%%
% UML diagrams go here
%%

\section{Data Dictionary}
The Data Dictionary sections will describe what each class does, along with a more detailed view of its
associations, attributes and methods that the class has.

%\include{}
%\include{}
%\include{}
%\include{}
%\include{}
%\include{}

\section{User Interface}
The User interface section will over the actions that a user will be able to take, explaining all menus, buttons,
pull down menus, selections, etc. This section will also describe actions that system administrators will need to
take to start and stop the application.

    \subsection{User Interaction}
    The user will interact with the application through the web interface. They will have several screens (views)
    to navigate through. Those will include the main screen to select the platform to download from (AWS, Dropbox,
    Google Drive, etc) and then will be presented with a view to enter their credentials. Once their credentials
    are entered they will be presented with the final view which contains their directory structure and contents
    and they will be able to multi-select which of those contents they would like to download.
    
      \subsubsection{Platform Selection Screen}
    
      \subsubsection{Authentication Screen}
    
      \subsubsection{File Selection Screen}
    
    \subsection{Administrator Interaction}
    The system administrator will interact with the application by using the builtin Django commands on the
    server that the application will run on. They will mainly only be starting and stopping the web application
    through this interface. Django does have a builtin web administration page but this application will not
    need to take advantage of this functionality.

      \subsubsection{Start Application}

      \subsubsection{Stop Application}

\section{Information Repositories}
The application uses several different information repositories. There will be one for each platform supported as
outlined in the \texttt{Requirements Specification} \cite{reqs}. For example, AWS storage will be an information
repository as well as Dropbox. The user's local hard drive will also be an information repository during
development and then once the client takes delivery their internal storage will become the information repository.

    \subsection{Cloud Platform Repository}
    The Cloud Repositories are the storage on the cloud for each platform that the user has files stored on.
    These files can be any type but will typically be an operating system ISO for the client's use case. These
    repositories can contain any arbitrary number of files and folders stored within this repository, the users
    will be able to nagivate these through the application and select which ones to download.
    
    \subsection{Local Storage or Internal Infrastructure}
    The second type of information repository that the application will contain is the location that the
    application will download the user selected files from. This will be the user's local storage during the
    development cycle and in production the client's internal infrastructure will be the repository for these
    files.

\newpage
\begin{thebibliography}{1}
\bibitem{reqs}
Author
\textit{title}.
[\textit{description}] year.
\end{thebibliography}

%\appendix
%\section{Appendix}

\end{document}
