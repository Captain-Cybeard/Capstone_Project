\subsection{class Platform\_template}
This class exists to allow future developers to add more cloud platforms to the product. Providing both convention for use, and layout 
for attribute creation and method creation. Included are basic methods that simplify the authentication. A Method to help format the 
API return data to comply to the standard specified. A Method to simplify the UI return to parse and complete the download for the user.
Other methods are used for helping simplify the class structure for future development. 

\subsubsection{Associations}
The association \textbf{UI Interface} is used when developing a new cloud platform class to be added into the application. This will also 
include any necessary inputs for the developer to adjust in order for the class to communicate with the UI. 


\subsubsection{Attributes}
\textbf{[Platform\_NAME]\_[METHOD]\_[PARAM]:[TYPE]} \\
This attribute is describes the convention for future developers to use when describing attributes to add to new classes. 
If an attribute is not inherent of a method or [METHOD] as described in the attribute, it is passed in with [NAME]\_\_[PARAM].
Helping to provide readability to future developers, and indicating how the attributes are being used through the application.

\textbf{[\_\_[Platform\_NAME]\_\_user\_download\_path:String} \\
This attribute needs to be included in every class, and will default to the Users C:/Downloads, unless specifically altered by the user 
through the application itself. 

\textbf{[\_\_[Platform\_NAME]\_get\_files\_list\_result:List} \\
This attribute is a recommendation to include in every class as a storage location for the return from the cloud API. This will be 
utilized in methods included within the template. 

\textbf{\_\_[Platform\_NAME]\_entries\_to\_download\_list:List} \\
This attribute is a recommendation to include in every class as storage for the return value of the UI for the cloud service. By using
this attribute, the developer can parse the return in the fashion to which they need to allow downloading the user specified data.

\textbf{[Platform\_NAME]\_format\_dict:Dictionary} \\
This attribute is a recommendation to include in every class to store the formated files into a folder/file dictionary that will allow 
users to view their files within included directories. Currently the application is not using the method to show folders for users.

\textbf{[Platform\_NAME]\_flat\_dict:Dictionary} \\
This attribute is a recommendation to include in every class to store a `flattened' Dictionary where only the files and their paths are 
stored. By using this attribute the user will only see their files without the folders that they are included with. 

\subsubsection{Methods}
The three following commands are used for any application that has a supported SDK for OAuth 2 authentication. If this is not present
within the platform being added, an authentication method will need to be created separately without the auth\_ methods.

\textbf{[Platform\_NAME]\_authentication\_flow(): In:Request} \\
This function call requires a redirect\_uri that may need to need to be specified within the app console 
of the platform being used. This class should return the Flow component of the OAuth2 authentication. This 
will likely need a request.session to help manage state.

\textbf{[Platform\_NAME]\_authentication\_start(): In:Request} \\
This function will initiate the starting parameters for the OAuth2 authentication to use this 
capability appropriately, after this method is created, this object will need to be instantiated
within the veiws.py script. Additionally the urls.py/urlpattern needs to be updated with the following: \\
  path('[Platform\_NAME]-auth-start', views.[Platform\_NAME].[Platform\_NAME]\_authentication\_start) \\
When the user selects the new plaform, the Views Index class will need to return a redirect function
redirect('[Platform\_NAME]-auth-start'). Finally this method should instantiate a redirect\_url to created by the 
authentication\_flow method noted previously. The return will need a django.shortcuts.redirect() to the redirect\_url 
to allow the user to log into the new platform.

\textbf{[Platform\_NAME]\_authentication\_finish(): In:Request} \\
This function will finalize the OAuth2 authentication. After this method is created, this object will need to be instantiated
within the veiws.py script. Additionally the urls.py/urlpattern needs to be updated with the following: \\
  path('[Platform\_NAME]-auth-finish', views.[Platform\_NAME].[Platform\_NAME]\_authentication\_finish) \\
The authentication\_flow, may require a redirect\_uri that includes the full uri to the [Platform\_NAME]-auth-finish
This function will catch any errors that arise within the OAuth2 authentication flow. 
This function will also call all necessary function to gather files and folders of the application users platform.
The returned object will then need to be formated into either a dict specified as: \\
  \{'path':'', 'dirs':[], 'files':[]\} \\
   In which every dir contains the same dict and dict list for subsequent directories
Currently the software requires a flat\_dict, in which all files are pulled out along with their full path and any required 
attributes necessary for downloading. The flat dict will be written as: \\
  \{'file':[]\} \\ 
with all the attributes required for downloading files.

\textbf{[NAME]\_get\_files\_list()} \\
This method is used to simplify the API usage for gathering the necessary metadata associated with cloud platform the class is being created for.
The return data should be saved local to this class in order to be used by future classes. 

\textbf{[NAME]\_format\_entries\_list():Object JSON} \\
This method is used to format the returned data provided by the [Platform\_NAME]\_get\_files\_list(self) method. The format includes a Dictionary defined as 
\begin{verbatim}
{
   "path":"",
   "dirs":[
      {
         "path":"",
         "dirs":[],
         "files":[]
      }
   ],
   "files":[
      {
         "path":"",
         "name":""
      }
   ]
}
\end{verbatim}
This Dictionary will be passed to the UI for the user to interact with when selecting what they want to download.

\textbf{[NAME]\_download\_selected\_entries(in list\_of\_selected\_downloads:Object JSON, [NAME]\_path:String):type} \\
This method is used to simplify the API usage for downloading. Once a user has selected the items they wish to download, and decided upon the 
download path. This function will use the cloud platform specified by the class to then download the list of items returned by the UI, into the 
specified user folder.

