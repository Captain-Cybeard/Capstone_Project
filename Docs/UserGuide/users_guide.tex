\documentclass{article}
\usepackage[margin=1.25in]{geometry}
\usepackage{fancyhdr}
\usepackage{graphicx}
\usepackage{vhistory}
\usepackage[parfill]{parskip}
\graphicspath{{../Images/}}

\renewcommand{\baselinestretch}{1.5}

% Set fancy looking header/footer and move page number to the right
\pagestyle{fancy}
\fancyhead{}
\fancyfoot{}
\fancyfoot[R]{\thepage}

\patchcmd{\section}{-3.5ex \@plus -1ex \@minus -.2ex}{-3.5ex \@plus -1ex \@minus -.2ex\setlength{\leftskip}{0cm}}{}{}
\patchcmd{\subsection}{-3.25ex\@plus -1ex \@minus -.2ex}{3.25ex\@plus -1ex \@minus -.2ex\setlength{\leftskip}{1cm}}{}{}
\patchcmd{\subsubsection}{1.5ex \@plus .2ex}{1.5ex \@plus .2ex\setlength{\leftskip}{2cm}}{}{}

\begin{document}
    \pagenumbering{gobble}
    \begin{titlepage}
    \begin{center}
        \vspace*{1cm}

        \Huge
        \textbf{Requirements Specification}

        \vspace{.5cm}
        \LARGE
        Captain Cybeard: Neil Before Us

        \vspace{1cm}

        \textbf{Ryan Breitenfeldt \textbar\ Noah Farris\\ Trevor Surface \textbar\ Kyle Thomas}

        \vspace{.2cm}
        \Large
        September 27, 2019

        \vspace{2cm}
        \includegraphics[scale=1]{logo}

        \vfill

        Washington State University Tri-Cities\\
        CptS 421 Software Design Project 1

    \end{center}
\end{titlepage}




    \tableofcontents
    \newpage
    \listoffigures


    \newpage
    \begin{versionhistory}
        \vhEntry{0.1}{04.20.2020}{KT}{Document Creation}
    \end{versionhistory}
    \newpage


    \pagenumbering{arabic}
    \section{Introduction}
        This document is an instruction manual for the users on how to use the \textit{Cloud Backup} software. There will be three parts.
        The first part will be how user's of application will interact with the application with the second part covering how system
        administrators will deploy the application. Followed by the third part, which is how developers will expand the application to
        include more cloud platforms.

    \section{Users}
      \subsection{Invocation}

      \subsection{Stopping}

      \subsection{Selecting a Cloud Provider}

      \subsection{Authenticating to the Cloud Platform}

      \subsection{Selecting and Downloading Files}

    \section{Administrators}
      \subsection{Starting the Application}

      \subsection{Stopping the Application}

    \section{Developers}
      \subsection{Adding a Platform}

      \subsubsection{Views}

      \subsubsection{Models}

\end{document}
